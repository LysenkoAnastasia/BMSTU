\documentclass[12pt]{article}
% Эта строка — комментарий, она не будет показана в выходном файле
\usepackage{ucs}
\usepackage[utf8]{inputenc} % Включаем поддержку UTF8
\usepackage[russian]{babel}  % Включаем пакет для поддержки русского языка
\title{Отчёт по численным методам}
\date{}
\author{}
 
\usepackage{geometry} % А4, примерно 28-31 строк(а) на странице	
    \geometry{paper=a4paper}
    \geometry{includehead=false} % Нет верх. колонтитула
    \geometry{includefoot=true}  % Есть номер страницы
    \geometry{bindingoffset=0mm} % Переплет    : 0  мм
    \geometry{top=20mm}          % Поле верхнее: 20 мм
    \geometry{bottom=25mm}       % Поле нижнее : 25 мм 
    \geometry{left=25mm}         % Поле левое  : 25 мм
    \geometry{right=25mm}        % Поле правое : 25 мм
    \geometry{headsep=10mm}  % От края до верх. колонтитула: 10 мм
    \geometry{footskip=20mm} % От края до нижн. колонтитула: 20 мм 
\usepackage{amsmath}    % \bar    (матрицы и проч. ...)
\usepackage{amsfonts}   % \mathbb (символ для множества действительных чисел и проч. ...)
\usepackage{mathtools}  % \abs, \norm
    \DeclarePairedDelimiter\abs{\lvert}{\rvert}
    \DeclarePairedDelimiter\norm{\lVert}{\rVert}
\usepackage{listings} %листинги

\usepackage{pgfplotstable}
\usepackage{pgfplots}

\pgfplotsset{compat=newest}
\usepgfplotslibrary{units}

\usepackage{xcolor}
\lstset { %
    language=C++,
    keywordstyle=\color{blue}\ttfamily,
    stringstyle=\color{red}\ttfamily,
    commentstyle=\color{green}\ttfamily,
    %backgroundcolor=\color{black!5}, % set backgroundcolor
    basicstyle=\footnotesize,% basic font setting
}

\begin{document}
    \newpage
    {
        \thispagestyle{empty}
        \centering
        
        \textbf{
        МОСКОВСКИЙ ГОСУДАРСТВЕННЫЙ ТЕХНИЧЕСКИЙ УНИВЕРСИТЕТ ИМЕНИ Н. Э. БАУМАНА \\
        Факультет информатики и систем управления \\
        Кафедра теоретической информатики и компьютерных технологий}
        \bigskip
        \bigskip
        \bigskip
        \bigskip
        \bigskip
        \bigskip
        \bigskip

        \vfill

        {\large Лабораторная работа №3}\\
        по курсу <<Численные методы>>\\
	\LARGE{<<Построение для таблично-заданной функции \\
			кубического сплайна,\\ 
			сплайна Акимы, \\
			Б-сплайна>>\\ }
	\normalsize

        \bigskip
        \vfill
        \hfill\parbox{5cm} {
            Выполнил:\\
            студент группы ИУ9-62 \hfill \\
            Беляев А. В.\hfill \medskip\\
            Проверила:\\
            Домрачева А. Б.\hfill
        }
        \vspace{\fill}

        
        Москва \number\year
        \clearpage
    }
	\newpage
	{
		\tableofcontents
		\clearpage
	}
    {
        \section{Постановка задачи}
    }
	
	Пусть имеется таблично-заданная функция $y_i=g(x_i), i=1..n$, а также функция $f(x)=ae^{bx}=2.477520 \times e^{0.232206 \times x}$, полученная из её аналитического представления в рамках Лабораторной работы №2.
	
	\begin{table}[h!]
\caption{$y_i = g(x_i)$ \label{table:one}}

\begin{center}
 \begin{tabular}{|c|c|c|c|c|c|c|c|c|c|c|} 
 \hline
X & 0.5 & 1.0& 1.5& 2.0& 2.5& 3.0& 3.5& 4.0& 4.5& 5.0\\
\hline
Y & 2.78& 3.13& 3.51& 3.94& 4.43& 4.97& 5.58& 6.27& 7.04& 7.91\\
\hline
\end{tabular}
\end{center}
\end{table}

	Необходимо:
	
		1) Построить кубический сплайн $S_i(x)$ для табличного представления функции y(x);
		
		2) Найти значения сплайн-функции в узлах интерполяции $x_i$, вычислить абсолютную погрешность;
		
		3) Нати значения сплайн-функции в точках между узлами интерполяции, вычислить абсолютную погрешность;
		
		4) Выполнить пункты 1 - 3 для для Б-сплайна;
		
		5) Выполнить пункт 4 для сплайна Акимы;
    
    	\clearpage
    
    	{
        	\section{Необходимые теоретические сведения}
   	}
    
    	{
    		\subsection{Кубические сплайны}
    	}
	
	Кубическим сплайном называется функция $S(x)$, которая:
	\begin{itemize}
		\item на каждом отрезке $[x_{i-1}, x_i]$ является многочленом степени не выше третьей
		\item дважды непрерывно дифференцируема на всем отрезке $[a, b]$
		\item в точках $x_i$ выполняется равенство $S(x_i)=f(x_i)$
	\end{itemize}
	
	Функция $S_i(x)$ представляется в следующем виде:
		$$ S_i(x) = a_i + b_i(x-x_{i-1})+c_i(x-x_{i-1})^2+d_i(x-x_{i-1})^3, x \in [x_{i-1}, x_i]$$
	Введем следующие обозначения: $h=\frac{x_n-x_0}{n}, y_i=y(x_i)$. Сплайн-функция удовлетворяет следующим условиям:
		$$S_i(x_i)=y_i;$$
		$$S_i'(x_i)=S_{i+1}'(x_i), i=1,2,...n-1;$$
		$$S_i''(x_i)=S_{i+1}''(x_i), i=1,2,...n-1;$$
		$$S_1'(x_0)=S_n'(x_n)=0.$$
	Из этих условий можно получить систему уравнений относительно коэффициентов $c_i$:
	$$
 \begin{cases}
   c_i=0, где i=0,n\\
   c_i + 4c_{i+1}+c_{i+2}=\frac{3}{h^2}(y_{i+1}-2y_i+y_{i-1}), где i=1,...,n-1
 \end{cases}$$
 
	Эту систему можно решить методом прогонки, а её решение использовать для получения остальных коэффициентов сплайна по следующим формулам:
		$$a_i = y_i;$$
		$$b_i=\frac{y_{i+1}-y_i}{h}-\frac{h}{3}(c_{i+1}+2c_i);$$
		$$d_i=\frac{c_{i+1}-c_i}{3h};$$
		
    
   	{
    		\subsection{Б - сплайны}
   	}
    
	Любая сплайн-функция заданной степени, гладкости и области определения может быть представлена, как линейная комбинация Б-сплайнов той же степени и гладкости на той же области определения. Так, кубический сплайн $S(x)$ на отрезке $[x_0, x_n]$ можно записать, как линейную комбинацию функций $B_k$:
	
		$$S(x)=\sum\limits_{k=-1}^{n+1} a_k B_k(x)$$
	
	Определим $B_k(x)=B_0(x-kh)$. Для интерполяции, проводимой на отрезке $[x_0, x_n]$, базовая функция $B_0$ определяется следующим образом:
	
	$$B_0(x)=
		\begin{cases}
			0 											& x \leq x_0-2h \\
			\frac{1}{6}(2h+(x-x_0))^3 						& x_0-2h < x \leq x_0-h \\
			\frac{2h^3}{3} - \frac{1}{2}(x-x_0)^2(2h+(x-x0))	 	& x_0-h < x \leq x_0 \\
			\frac{2h^3}{3} - \frac{1}{2}(x-x_0)^2(2h-(x-x0))	 	& x_0 < x \leq x_0 + h \\
			\frac{1}{6}(2h-(x-x_0))^3 						& x_0+h < x \leq x_0 + 2h\\
    			0											& x_0 + 2h < x 
		\end{cases}
	$$
	
	где $h=\frac{x_n-x_0}{n}$ -- это расстояние между узлами.
	
	Значение функции $S(x_k)$ узловой точки принимает ненулевое значение только для $B_{k-1}, B_k, B_{k+1}, B_{k+2}$:
	
		$$S(x_k)=a_{k-1}B_{k-1}(x_k) + a_k B_k (x_k) + a_{k+1}B_{k+1}(x_k) + a_{k+2}B_{k+2}(x_k) =f(x_k)$$
	
	Необходимо определить коээфициенты $a_i$. Для этого из определений $B_k(x_k)$ и $B_0(x_0)$:
	
		$$B_{k-1}(x_k)=B_0(x_0+h)=\frac{h^3}{6}$$
		$$B_k(x_k)=B_0(x_0)=\frac{2h^3}{3}$$
		$$B_{k+1}(x_k)=B_0(x_0-h)=\frac{h^3}{6}$$
		$$B_{k+2}(x_k)=B_0(x_0-2h)=0$$
		
	подставим значения $B_{k-1}, B_k, B_{k+1}, B_{k+2}$ в функцию $S(x_k)$ и получим рекуррентное соотношение для коэффициентов $a_i$:
	
		$$a_{k-1}+4a_k+a_{k+1}=\frac{6}{h^3}f(x_k), k = 0,...,n$$
		
	Рассмотрим $S''_3(x)$:
		
		$$S''_3(x)=\sum\limits_{k=-1}^{n+1} a_k B''_k(x)$$	
	
	Дважды дифференцируя $B-0$, получим:
	
	$$B''_0(x)=
		\begin{cases}
			0 											& x \leq x_0-2h \\
			2h+(x-x_0)			 						& x_0-2h < x \leq x_0-h \\
			-2h-3(x-x0))								 	& x_0-h < x \leq x_0 \\
			-2h+3(x-x_0)								 	& x_0 < x \leq x_0 + h \\
			2h-(x-x0)										& x_0+h < x \leq x_0 + 2h\\
    			0											& x_0 + 2h < x 
		\end{cases}
	$$
	
	Тогда, приравняв нулю $S''_3(x_0)$, прийдем к равенству $a_{-1}-2a_0+a_1=0$.
	Определим рекуррентное соотношение для $k=0$:
	
		$$a_{-1}+4a_0+a_{k+1}=\frac{6}{h^3}f(x_k)$$
	
	Вычтем из этого соотношения $a_{-1}-2a_0+a_1=0$ и получим значения для $a_0$:
	
		$$a_0=\frac{1}{h^3}f(x_0)$$
	
	По аналогии находим значение $a_n$:
	
		$$a_n=\frac{1}{h^3}f(x_n)$$
		
	Значения всех остальных коэффициентов $a_i$ вычисляем из системы вида:
\begin{equation*}
	\begin{pmatrix}
		1 & 0 & 0 & 0 & 0 & \dots & 0 \\
		1 & 4 & 1 & 0 & 0 & \dots & 0 \\
		0 & 1 & 4 & 1 & 0 & \dots & 0 \\
		\vdots & \vdots & \vdots & \ddots & \dots & \dots & \vdots \\
		0 & \dots & 0 & 1 & 4 & 1 & 0 \\
		0 & \dots & 0 & 0 & 1 & 4 & 1 \\
		0 & \dots & 0 & 0 & 0 & 0 & 1 
	\end{pmatrix}
	\begin{pmatrix}
		a_0 \\
		a_1 \\
		a_2 \\
		\vdots \\
		a_{n-2} \\
		a_{n-1} \\
		a_n
	\end{pmatrix}
	= \frac{1}{h^3}
	\begin{pmatrix}
		f(x_0) \\
		6f(x_1) \\
		6f(x_2) \\
		\vdots \\
		6f(x_{n-2}) \\
		6f(x_{n-1}) \\
		f(x_n)
	\end{pmatrix}
\end{equation*}
	
	и дополнительно определим $a_{-1}$ и $a_{n+1}$:
	
		$$a_{-1}=2a_0-a_1$$
		$$a_{n+1}=2a_n-a_{n-1}$$
    	{
    		\subsection{Сплайны Акимы}
    	}
    
	Сплайн Акимы – это особый вид сплайна, предложенный Хироши Акимой, устойчивый к выбросам. Недостатком кубических сплайнов является то, что они склонны осциллировать в окрестностях точки, существенно отличающейся от своих соседей.
	
	Свойством сплайна Акимы является его локальность – значения функции на отрезке $[x_i, x_{i+1}]$ зависят только от значений $f_{i-2}, f_{i-1}, f_i, f_{i+1}, f_{i+2}$, вследствие чего на отрезках, граничащих с выбросом, практически отсутствуют признаки осцилляции. \\
	
	Сплайн-функцию можно представить в следующем виде:
	
		$$ S(x) = c_1 + c_2(x-x_{i})+c_3(x-x_{i})^2+c_4(x-x_{i})^3$$ 
	
	где $x \in [x_i, x_{i+1}], i \in [1, k]$, а коэффициенты $c_i$ вычисляются подобным образом:
	
		$$c_1 = y_i$$
		$$c_2 = m_i$$
		$$c_3 = \frac{3(m_i - t_i) + t_{i+1} - t_i}{h_i}$$
		$$c_4 = \frac{\frac{t_{i+1}-t_i}{h_i}-2(m_i-t_i)}{{h_i}^2}$$
	
	где $m$ является первой производной узла и вычисялется следующим образом:
	
		$$m_{i}=\frac{f_{i+1}-f_i}{x_{i+1}-x_i}, i \in [0; n)$$
		
	 и где $f_i=f(x_i)$, $h_i=x_{i+1}-x_i$ и $m_i=\frac{y_{i+1}-y_i}{h_i}$.
		
	Значения на концах отрезка функции вычисляются отдельно согласно формулам:
	
		$$m_{-2}=3m_0-2m_1$$
		$$m_{-1}=2m_0-m_1$$
		$$m_{n+1}=2m_{n}-m_{n-1}$$
		$$m_{n+2}=3m_{n}-2m_{n-1}$$
		
	
	
	
	
	
    
    {
        \section{Текст программы}
    }
    
    Для написания программы был использован язык C++.
    
    {
        \lstset{basicstyle=\tiny}
        \begin{lstlisting}

#define xtype float
#define N 10

xtype _a = 2.477520;
xtype _b = 0.232206;

xtype xs[] = {0.5, 1.0, 1.5, 2.0, 2.5, 3.0, 3.5, 4.0, 4.5, 5.0, 5.5};
xtype ys[] = {2.78, 3.13, 3.51, 3.94, 4.43, 4.97, 5.58, 6.27, 7.04, 7.91, 8.89};
xtype xsNew[] = {0.50, 0.75, 1.00, 1.25, 1.50, 1.75, 2.00, 2.25, 2.50, 2.75, 3.00, 3.25, 3.50, 3.75, 4.00, 4.25, 4.50, 4.75, 5.00, 5.25, 5.50};
xtype ysNew[] = {2.78, 2.95, 3.13, 3.31, 3.51, 3.72, 3.94, 4.18, 4.43, 4.69, 4.97, 5.27, 5.58, 5.92, 6.27, 6.65, 7.04, 7.47, 7.91, 8.38, 8.89};
xtype xsNew1[2*N+1];
xtype ysNew1[2*N+1];
xtype f(xtype x) { return _a * exp(_b*x); }

xtype h = xs[1] - xs[0];

typedef struct spline {
    xtype a[N];
    xtype b[N];
    xtype c[N];
    xtype d[N];
} _spline;

typedef struct bispline {
    xtype x[N+3];
} _bispline;

typedef struct akame {
    xtype a[N+4];
    xtype b[N+4];
    xtype c[N+4];
    xtype d[N+4];
} _akame;

spline populateSpline(int n) {

    xtype ak[n-3], bk[n-2], ck[n-3], dk[n-2], ek[n];
    xtype a[n], b[n], c[n], d[n];

    xtype h = xs[1] - xs[0];

    for (int i = 0; i < n - 2; i++) {

        if (i < n-3) ak[i] = ck[i] = 1;

        bk[i] = 4;
        dk[i] = 3 / (h * h) * (ys[i+2] - 2*ys[i+1] + ys[i]);
    }

    solveMatrix(n-2, ak, bk, ck, dk, ek);

    for (int i = 1; i < n-1; i++) c[i] = ek[i-1];
    c[0] = c[n-1] = 0;
    
    for (int i = 0; i < n-1; i++) {
        a[i] = ys[i];
        b[i] = (ys[i+1] - ys[i]) / h - (h/3) * (c[i+1] + 2*c[i]);
        d[i] = (c[i+1] - c[i]) / (3*h);
    }


    spline spZ;

    for (int i = 0; i < n; i++) {
        spZ.a[i] = a[i];
        spZ.b[i] = b[i];
        spZ.c[i] = c[i];
        spZ.d[i] = d[i];
    }
    return spZ;
}

xtype countSplineAt(_spline spline, xtype x) {
    //S_j[x] = a_j + b_j(x-xj) + c_j(x-xj)^2 + d_j(x-xj)^3
    int i;
    for (i = 0; i < N; ++i)
        if (x >= xs[i] && x <= xs[i+1])
            break;

    return  spline.a[i] +
            spline.b[i]*(x - xs[i]) +
            spline.c[i]*(x - xs[i])*(x - xs[i]) +
            spline.d[i]*(x - xs[i])*(x - xs[i])*(x - xs[i]);
}


bispline populateBiSpline(int len) {

    int size = len;
    int n = size - 1;
    xtype a[size], b[size], c[size], d[size], e[size+2];

    for (int i = 1; i < n; i++) {
        a[i] = 1;
        b[i] = 4;
        c[i] = 1;
        d[i] = 6*ys[i];
    }

    a[0] = 0;		a[n] = 0;
    b[0] = 1;		b[n] = 1;
    c[0] = 0;		c[n] = 0;
    d[0] = ys[0];	d[n] = ys[n-1];

    for (int i = 0; i < n; i++)  d[i] /= (h*h*h);

    solveMatrixNew(len, a, b, c, d, e);


    _bispline bsp;

    for (int i = 0; i < size+2; i++)
        bsp.x[i] = e[i];

    return bsp;
}

xtype baseX(int k, xtype x, xtype kh) {
    if (0 == k) {

        xtype xz = xs[0];

        if (xz + 2*h <= x) return 0;
        if (xz + h <= x && x < xz + 2*h) return pow(2*h + xz - kh - (x - kh), 3)/6;
        if (xz <= x && x < xz + h)
            return 2*pow(h, 3)/3 - pow(-xz + kh + (x - kh), 2)*(2*h + xz - kh - (x - kh))/2;
        if (xz - h <= x && x < xz)
            return 2*pow(h, 3)/3 - pow(-xz + kh + (x - kh), 2)*(2*h - xz + kh + (x - kh))/2;
        if (xz - 2*h <= x && x < xz - h) return pow(2*h - xz + kh + (x - kh), 3)/6;
        if (x < xz - 2*h) return 0;

    } else {
        return baseX(0, x - k*h, -k*h);
    }

}

xtype countBiSplineAt(_bispline bsp, xtype val) {
    int i;
    for (i = 0; i < N+2; ++i)
        if (xs[i] <= val && val <= xs[i+1])
            break;

    return  bsp.x[i + 0] * baseX(i - 1, val, 0) +
               bsp.x[i + 1] * baseX(i + 0, val, 0) +
               bsp.x[i + 2] * baseX(i + 1, val, 0) +
               bsp.x[i + 3] * baseX(i + 2, val, 0);
}


akame populateAkame(int n) {
    int i;
    xtype m[n + 4];
    
    //-2;-1;    0.. n-1    ;n;n+1
    for (i = 2; i < n + 2; i++) 
        m[i] = (ys[i+1-2] - ys[i-2]) / (xs[i+1-2] - xs[i-2]);

    m[0] = 3*m[2] - 2*m[3]; //m[-2]=m[0]-m[1]
    m[1] = 2*m[2] - m[3];
    m[n+2] = 2*m[n+1] - 2*m[n];
    m[n+3] = 3*m[n];

    xtype ne, alpha_i, h_i;
    xtype t_l[n], t_r[n];

    for (i = 2; i < n + 2; i++) {
        ne = fabs(m[i+1] - m[i]) + fabs(m[i-1] - m[i-2]);

        if (ne > 0) {
            alpha_i = fabs(m[i-1] - m[i-2]) / ne;
            t_l[i] = m[i-1] + alpha_i*(m[i] - m[i-1]);
            t_r[i] = t_l[i];
            
        } else {
            t_l[i] = m[i-1];
            t_r[i] = m[i];
        }
    }


    akame ga_spline;
    
    for (i = 2; i < n+2; i++) {
        h_i = xs[i+1-2] - xs[i-2];

        ga_spline.a[i] = ys[i-2];
        ga_spline.b[i] = t_r[i];
        ga_spline.c[i] = (3*m[i] - 2*t_r[i] - t_l[i+1]) / h_i;
        ga_spline.d[i] = (t_r[i] + t_l[i+1] - 2*m[i]) / (h_i*h_i);
    }
    return ga_spline;
}

xtype countAkameAt(akame ak, xtype x, int i) {

    for (i = 0; i < N; ++i)
        if (x >= xs[i] && x <= xs[i+1])
            break;

    return  ak.a[i+2] +
            ak.b[i+2]*(x - xs[i]) +
            ak.c[i+2]*(x - xs[i])*(x - xs[i]) +
            ak.d[i+2]*(x - xs[i])*(x - xs[i])*(x - xs[i]);
}

int main()
{
    for (int i = 0; i <= 2*N; i++)  xsNew1[i] = (i % 2) == 0 ? xsNew[i] : xsNew[i-1] + 0.15;
    for (int i = 0; i <= 2*N; i++) ysNew1[i] = (i % 2) == 0 ? ysNew[i] : f(xsNew1[i]);
    
    printf("populating Spline\n");
    spline sp = populateSpline(N+1);

    printf("populating Bispline\n");
    bispline bs = populateBiSpline(N+2);

    printf("populating Akame ga Spline\n");
    akame as = populateAkame(N);

    for ( int i = 0; i <= 2*N; i++) {
        xtype sVal = countSplineAt(sp, xsNew[i]);
        xtype bsVal = countBiSplineAt(bs, xsNew[i]);
        xtype asVal = countAkameAt(as, xsNew[i], i);

        printf("x[%d]=%.2f y=%.5f \ts=%.5f d=%.5f \tbi=%.5f d=%.5f \ta=%.5f d=%.5f\n",
               i, xsNew[i], ysNew[i],
               sVal, fabs(ysNew[i] - sVal),
               bsVal, fabs(ysNew[i] - bsVal),
               asVal, fabs(ysNew[i] - asVal));
    }
    return 0;
}
        \end{lstlisting}
    }
    
    \clearpage


\newpage
    {
        \section{Результаты}
    }
    Обозначим значения кубических сплайнов, как CubicSpline или C. Значения Б-сплайнов, как BasisSpline или B. Значения сплайнов Акимы, как AkimaSpline или A. Проведем тестирование. В результате работы программы были получены следующие результаты в узлах интерполяции:
    
    \begin{table}[h]
    \caption{Значения в узлах интерполяции}
	\begin{center}
 	\begin{tabular}{|c|c|c|c|c|} 
\hline
x & f(x) & CubicSpline(x) & BasisSpline(x) & AkimaSpline(x) \\
\hline
0.50 & 2.780000 & 2.780000 & 2.780000 & 2.780000 \\
\hline
1.00 & 3.130000 & 3.130000 & 3.130000 & 3.130000 \\
\hline
1.50 & 3.510000 & 3.510000 & 3.510000 & 3.510000 \\
\hline
2.00 & 3.940000 & 3.940000 & 3.940000 & 3.940000 \\
\hline
2.50 & 4.430000 & 4.430000 & 4.430000 & 4.430000 \\
\hline
3.00 & 4.970000 & 4.970000 & 4.970000 & 4.970000 \\
\hline
3.50 & 5.580000 & 5.580000 & 5.580000 & 5.580000 \\
\hline
4.00 & 6.270000 & 6.270000 & 6.270000 & 6.270000 \\
\hline
4.50 & 7.040000 & 7.040000 & 7.040000 & 7.040000 \\
\hline
5.00 & 7.910000 & 7.910000 & 7.910000 & 7.910000 \\
\hline

\end{tabular}
\end{center}
    \end{table}

	Как следует из результатов тестирования, значения всех сплайн-функций полностью совпадают между собой и с ``ожидаемым'' значением в узлах интерполяции. Это говорит о корректности построения сплайн-функций.
	
	Теперь вычислим значения между узлами интерполяции в точках $x_{new}=x_i+0.15$:

\begin{table}[h]
    \caption{Значения между узлами интерполяции}
	\begin{center}
 	\begin{tabular}{|c|c|c|c|c|c|c|c|}
\hline
x & f(x) & C(x) & $\abs{f(x)-C(x)}$ & B(x) & $\abs{f(x)-B(x)}$ & A(x) & $\abs{f(x)-A(x)}$ \\
\hline
0.65 & 2.881157 & 2.882504 & 0.001347 & 2.883517 & 0.002360 & 2.882086 & 0.000929 \\
\hline
1.15 & 3.235861 & 3.239793 & 0.003932 & 3.239804 & 0.003943 & 3.240194 & 0.004333 \\
\hline
1.65 & 3.634233 & 3.632949 & 0.001283 & 3.632908 & 0.001325 & 3.632210 & 0.002023 \\
\hline
2.15 & 4.081649 & 4.081181 & 0.000468 & 4.081334 & 0.000325 & 4.081136 & 0.000513 \\
\hline
2.65 & 4.584146 & 4.586258 & 0.002112 & 4.585686 & 0.001543 & 4.586346 & 0.002200 \\
\hline
3.15 & 5.148508 & 5.144826 & 0.003682 & 5.146964 & 0.001354 & 5.144316 & 0.004192 \\
\hline
3.65 & 5.782349 & 5.778608 & 0.003741 & 5.769631 & 0.012718 & 5.778488 & 0.003861 \\
\hline
4.15 & 6.494223 & 6.492341 & 0.001882 & 6.467652 & 0.026571 & 6.491814 & 0.002409 \\
\hline
4.65 & 7.293736 & 7.288169 & 0.005568 & 7.246946 & 0.046790 & 7.289189 & 0.004547 \\
\hline
5.15 & 8.191680 & 8.194182 & 0.002502 & 8.108945 & 0.082735 & 8.14436 & 0.047320 \\
\hline
\end{tabular}
\end{center}
    \end{table}
    
    Как видно из таблицы результатов выше, Б-сплайны и сплайны Акимы дают погрешность примерно одного порядка, что на порядок меньше, чем погрешность кубического сплайна.
        
\begin{tikzpicture}
\begin{axis}[
height=300pt,
width=430pt]
\addplot table [y=P, x=$F$]{data.txt};
\addlegendentry{$f(x)=ae^{bx}$}
\addplot table [y=P, x=$C$]{data.txt};
\addlegendentry{$CubicSpline$}
\end{axis}
\end{tikzpicture}

\begin{tikzpicture}
\begin{axis}[
height=300pt,
width=430pt]
\addplot table [y=P, x=$F$]{data.txt};
\addlegendentry{$f(x)=ae^{bx}$}
\addplot table [y=P, x=$B$]{data.txt};
\addlegendentry{$BasisSpline$}
\addplot table [y=P, x=$A$]{data.txt};
\addlegendentry{$AkimaSpline$}
\end{axis}
\end{tikzpicture}

	Выше приведены графики для наглядного сравнения результатов. Заметно отсутствие ``выбросов'' значений (нужен больший масштаб чтобы полностью оценить различия результатов) а также заметно ``перекрытие'' графика исходной функции графиками Б-сплайна и сплайна Акимы, что наглядно демонстрирует значения их погрешностей по таблице.



    {
        \section{Выводы}
        
    }
	В ходе работы были реализованы построения кубических сплайнов, Б-сплайнов и сплайнов Акимы по таблично-заданной функции. В вычислениях присутствует методологическая погрешность. Было вычислено ее абсолютное значение для данных методов Так, было выявлено, что погрешность кубического сплайна не превышает $0.006$, погрешность Б-сплайна выше, но не превосходит $0.083$ и погрешность сплайна Акимы не выше $0.048$. Отсюда следует, что наилучший результат дает кубический сплайн. Результаты Б-сплайна и сплайна Акимы примерно одинаковы и оба уступают кубическом сплайну. Отсюда следует вывод, что кубический сплайн больше подходит для решения реальных практических задач, нежели остальные методы.
    
   
    \end{document}
    
